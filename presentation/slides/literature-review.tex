\mode*
\begin{frame}
  \section{Literature Review}
  \frametitle{Literature Review}
  \begin{itemize}
    \item The authors of \cite{indian} developed a similar approach
      \begin{itemize}
        \item Same dataset
        \item Only Na\"ive Bayes and Decision Tree
        \item Feature selection based on subjective choice
        \item No details disclosed regarding their method of assigning the
          class labels
      \end{itemize}
    \item A different approach is illustrated in \cite{forecast}
      \begin{itemize}
        \item Goal: predict crime hotspots in the US
        \item They first mapped unique crime records to the
          corresponding locations of the US using geographical coordinates
        \item Their next step was to find clusters
        \item The prediction of crime activity within the clusters was
          performed by an artificial neural network
      \end{itemize}
  \end{itemize}
\end{frame}

A similar approach to ours was already developed by the authors of
\cite{indian} who employed "WEKA" for predicting the crime category
using the same data as well. In order to do this, they selected a
different set of features which is omitted here for the sake of saving
space. Their method of selection as well as the assignment of the
class labels "High", "Medium" and "Low" to each instance in the
dataset is -- countrary to our approach -- solely based on subjective
choice.

In order to perform the classification task, they chose two
classifiers, namely a Na\"ive Bayesian and a decision tree. They did
however use the same evaluation metrics as used in this paper, which
allows for comparison of the different approaches.\\


\noindent In contrast to those ways of predicting crime there is a
different approach illustrated in \cite{forecast}. In order to predict
crime in the future they analyzed a dataset which contains the
following attributes:
\begin{itemize}
	\setlength{\itemsep}{-2pt}
	\item Time
	\item Day
	\item Month
	\item Weather
	\item Location(geographical coordinates) 
\end{itemize}
The records in the dataset are geographically mapped onto the certain
regions of the United States of America. Using their own
implementation of a cluster analysis they detect hot-spots of crime
according to the given features. Afterwards an artificial neural
network tries to forecast crime activity in the future.


\mode<all>
