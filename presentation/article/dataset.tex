\section{Dataset}

In the course of this research we use real crime data which can be obtained from Machine Learning Repository\footnote{http://archive.ics.uci.edu/ml/datasets/Communities+and+Crime}.

From the 128 attribute the dataset provides, only the most significant (continuous) features are chosen:

\begin{itemize}
	\setlength{\itemsep}{-2pt}
	\item population 
	\item mean people per household
	\item percentage of population that is african american
	\item percentage of population that is caucasian
	\item percentage of population that is of asian heritage 
	\item percentage of population that is of hispanic heritage
	\item percentage of population that is 16-24 in age
	\item percentage of population that is 65 and over in age 
	\item median household income
	\item percentage of people under the poverty level
	\item percentage of people 25 and over with less than a 9th grade education 
	\item percentage of people 16 and over, in the labor force, and unemployed 
	\item percentage of population who are divorced
	\item percent of people who do not speak English well 
	\item police officers per 100K population
	\item population density in persons per square mile
	\item total number of violent crimes per 100K popuation 
\end{itemize}

The goal is to predict the crime category, which is an \textit{artificial}\footnote{this attribute is added by us}, discrete, ordinal attribute. The values of this attributes, i.e. the classes we are trying to predict,  are obtained by clustering the attribute "total number of violent crimes per 100K population" into three labels: "high", "medium" and "low". To accomplish this we use an unsupervised learning approach. 

For predicting the "crime category", the attribute "total number of violent crimes per 100K population" is removed from the dataset, because it is already represented by the class division and would harm the ability of the classifier to draw conclusions from other relevant attributes.