\section{Dataset}

This research is based on real crime data which can be obtained from the Machine Learning
Repository\footnote{http://archive.ics.uci.edu/ml/datasets/Communities+and+Crime}. The
dataset contains 1994 instances. The whole set consists of a union of
three subsets, namely the 1990 US Census, law enforcement data from
the 1990 US LEMAS survey and crime data from the 1995 FBI UCR.

From the 128 attributes the dataset provides, only the most
significant features are chosen.
The exact procedure that lead to this subset of features is
described in a more detailed manner in \fref{sec:feature_selection}.
All chosen attributes (except, of course, for the class label)
are continuous and normalized, no attribute has any missing
values.
A list of the attributes that will be used for classification is
presented here:
\begin{itemize}
  \setlength{\itemsep}{-2pt}
  \item percentage of population that is african american
  \item percentage of population that is caucasian
  \item median household income
  \item percentage of households with investment / rent income in 1989
  \item percentage of households with public assistance income in 1989
  \item percentage of people under the poverty level
  \item percentage of people 25 and over with less than a 9th grade education
  \item percentage of people 16 and over, in the labor force, and unemployed
  \item percentage of population who are divorced
  \item percentage of kids in family housing with two parents
  \item percentage of kids born to never married
  \item percentage of people who do not speak English well
  \item percentage of people in owner occupied households
  \item crime category (the class to be predicted, eigher "High", "Medium" or "Low")
\end{itemize}

\noindent It should be noted that the class attribute "crime category" is an \textit{artificial}\footnote{this attribute is added by us}
attribute, which is based on the attribute "total number of violent
crimes per 100K population". This attribute, which is continuous in
nature and therefore does not lend itself for classification, is
essentially grouped into three classes ("High", "Medium" or "Low")
based on its value. The details of this approach are outlined in \fref{sec:assignment}.

%% The goal is to predict the crime category, which is an
%% \textit{artificial}\footnote{this attribute is added by us}, discrete and ordinal attribute.
%% The values of this attributes, i.e.
%% the classes that are to be predicted, are obtained by clustering the attribute "total number of violent
%% crimes per 100K population" into three labels: "high", "medium" and "low". To accomplish this, we use an
%% unsupervised learning approach, namely k-Means clustering.

%% For predicting the "crime category", the attribute "total number of violent crimes per 100K population" is
%% removed from the dataset, because it is already represented by the class division and would harm the ability
%% of the classifier to draw conclusions from other relevant attributes.
