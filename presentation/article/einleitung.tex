\section{Introduction}

In a time of limited assets regarding fighting crimes,
we aim to provide additional guidance to
those distributing the resources by training machine learning
classifiers to predict the criminal category of different cities.

After selecting the features of a city that have the most striking
influence on its criminal category (i.e. high crime rate, medium crime
rate or low crime rate), the classifiers are trained to learn the
complex relations between these characteristics and the corresponding
category. The different classifiers used for this task are
namely k-Nearest Neighbors, Na\"ive Bayes, a Decision Tree and a
Neural Network.

Each classifier is first trained on its own, the obtained models are
then combined into an ensemble to merge the individual
strengths of each classifier. The decision of the ensemble is found by
conducting a majority voting.
The metrics accuracy, precision, recall and f-measure are
used to evaluate the results.

In the course of this research the open source data mining software
WEKA is utilized to apply the algorithms to the dataset.

