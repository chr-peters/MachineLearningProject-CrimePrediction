\section{Introduction}

To recognize possible crimes of potential criminals we analyze the following criminal statistics from 2016 by the Federal Criminal Police Office of Germany (BKA):

\vspace*{-0.2cm}

\begin{itemize}
	\setlength{\itemsep}{-2pt}
	\item main table
	\item suspects divided by age and gender
	\item other information about the suspects
\end{itemize}
In order to solve this problem, we first examine the following features of the candidate:  

\vspace*{-0.2cm}

\begin{itemize}
	\setlength{\itemsep}{-2pt}
	\item gender
	\item age
	\item foreigner
	\item population of residence
	\item drug consumption
	\item gun ownership
	\item former suspect
\end{itemize}
Based on this data, the program predicts the probability of the crimes the person is most likely to commit. To achieve this we employ the k-Nearest Neighbor approach as well as a neural network.   

It is important to emphasize that this program can only be reasonably applied to a context where other signs already indicate that the given person is presumably a suspect.