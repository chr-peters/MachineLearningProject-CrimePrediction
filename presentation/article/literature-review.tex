\section{Literature Review}
A similar approach to ours was already developed by the authors of \cite{indian} who employed "WEKA" for predicting the crime category using the same data as well. In order to do this, they extracted the following features:

\begin{itemize}
	\setlength{\itemsep}{-2pt}
	\item State
	\item Population
	\item Median household income
	\item Median family income
	\item Per capita income
	\item Num. of people under poverty level
	\item Percentage of people 25 and over with less than a 9th grade education
	\item Percentage of people 25 and over that are not high school graduates
	\item Percentage of people 25 and over with a bachelor’s degree or higher education
	\item Percentage of people 16 and over in the labor force, and unemployed
	\item Percentage of people 16 and over who are employed
	\item Total number of violent crimes per 100K population
\end{itemize}

Unlike us they did not use unsupervised clustering to determine the division of the crime category into "high", "medium" and "low", but they set the relevant boundaries themselves. 
While they only used a decision tree and a Naive Bayes Classifier to predict the class, we use an ensemble of different classifiers, which also includes the ones they choose. 
To measure the results we will use the same metrics. This way we will be able to compare the different results. \\


\noindent In contrast to those approaches there is a different approach illustrated in \cite{forecast}. In order to predict crime in the future they analyze a dataset which contains the following attributes:
\begin{itemize}
	\setlength{\itemsep}{-2pt}
	\item Time
	\item Day
	\item Month
	\item Weather
	\item Location(geographical coordinates) 
\end{itemize}
The records in the dataset are geographically mapped onto the certain regions of the United States of America. Using their own implementation of a cluster analysis they detect hot-spots of crime according to the given features. Afterwards an artificial neural network tries to forecast crime activity in the future. 

\vspace*{\fill}