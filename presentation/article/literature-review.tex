\section{Literature Review}
A similar approach to ours was already developed by the authors of
\cite{indian} who employed WEKA for predicting the crime category
using the same data as well. In order to do this, they selected a
different set of features which is omitted here for the sake of saving
space. Their method of selection as well as the assignment of the
class labels "High", "Medium" and "Low" to each instance in the
dataset is -- countrary to our approach -- solely based on subjective
choice.

In order to perform the classification task, they chose two
classifiers, namely a Na\"ive Bayesian and a decision tree. They did
however use the same evaluation metrics as used in this paper, which
allows for comparison of the different approaches.\\

%% \begin{itemize}
%% 	\setlength{\itemsep}{-2pt}
%% 	\item state
%% 	\item population
%% 	\item median household income
%% 	\item median family income
%% 	\item per capita income
%% 	\item num. of people under poverty level
%% 	\item percentage of people 25 and over with less than a 9th grade education
%% 	\item percentage of people 25 and over that are not high school graduates
%% 	\item percentage of people 25 and over with a bachelor’s degree or higher education
%% 	\item percentage of people 16 and over in the labor force, and unemployed
%% 	\item percentage of people 16 and over who are employed
%% 	\item crime category ("High", "Medium" or "Low", attribute to
%%           be predicted)
%% \end{itemize}

%% Unlike us they did not use unsupervised clustering to determine the
%% division of the crime category into "high", "medium" and "low", but
%% they set the relevant boundaries themselves.
%% While they only used a decision tree and a Naive Bayes Classifier to
%% predict the class, we use an ensemble of different classifiers, which
%% also includes the ones they choose.
%% To measure the results we will use the same metrics,
%% this way we will be able to compare the different results. \\

\noindent In contrast to those ways of predicting crime there is a
different approach illustrated in \cite{forecast}. In order to predict
crime in the future they analyzed a dataset which contains the
following attributes:
\begin{itemize}
	\setlength{\itemsep}{-2pt}
	\item Time
	\item Day
	\item Month
	\item Weather
	\item Location(geographical coordinates) 
\end{itemize}
The records in the dataset are geographically mapped onto the certain
regions of the United States of America. Using their own
implementation of a cluster analysis they detect hot-spots of crime
according to the given features. Afterwards an artificial neural
network tries to forecast crime activity in the future.

